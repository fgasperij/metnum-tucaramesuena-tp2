\subsection{Comparación de la complejidad de los métodos}
El método 1 trabaja con la matriz $AA^t$ y el método 2 trabaja con la matriz $A^tA$.
La matriz $A\in\mathbb{R}^{n \times m}$ con $m = pixels(img)$ y $n = nimgp \times \#personas$. Por lo tanto como $A^t\in\mathbb{R}^{m \times n}$ nos
queda que $AA^t\in\mathbb{R}^{n \times n}$ y $A^tA\in\mathbb{R}^{m \times m}$.

En el \'unico punto que divergen los algoritmos es en el c\'alculo de los $k$ mayores autovalores, y sus
autovectores asociados, de $M_x = A^tA$, la matriz de covarianza. 
El m\'etodo 1 aplica el m\'etodo de la potencia con deflaci\'on directamente sobre la matriz $M_x$. 
El m\'etodo 2 primero aplica el m\'etodo de la potencia sobre la matriz $M^t_x = AA^t$ y después realiza un despeje para obtener los 
autovalores y autovectores de $M_x$. 
Por lo tanto tenemos que los dos métodos utilizan la misma función para calcular autovectores y autovalores:
\newpage
\begin{algorithm}
\caption{calcularAutovaloresYAutovectores(Matriz $A$)}
\label{pseudo:calcularAutovaloresYAutovectores}
\begin{algorithmic}
\FOR {$i = 1$ hasta $k$}
  \STATE [$v_i$, $\lambda_i$] = metodoDeLaPotencia($A$)
  \STATE $A$ = deflacionar($A$, $v_i$, $\lambda_i$)
\ENDFOR
\end{algorithmic}
\end{algorithm}
y el método 2 además utiliza una función para despejar:
\begin{algorithm}
\caption{despejar(Matriz $A$, Vector $v$, Double $\lambda$)}
\label{pseudo:despejar}
\begin{algorithmic}
  \STATE $u$ = $v\frac{ \sqrt{\lambda_{i} } }{ \lambda_{i} }  A^{t}$   
\end{algorithmic}
\end{algorithm}
por lo tanto la complejidad del método 1 es (abusando de la notación asintótica):
\begin{displaymath}
  O(metodo1) = k (O(metodoDeLaPotencia) + O(deflacionar))
\end{displaymath}
y del método 2:
\begin{displaymath}
  O(metodo2) = k (O(metodoDeLaPotencia) + O(deflacionar)) + k O(despejar)
\end{displaymath}
La complejidad del método de la potencia es:
\begin{displaymath}
  O(metodoDeLaPotencia) = iter (O(Ax) + O(\frac{y}{{\parallel}y{\parallel}}))
\end{displaymath}
$A\in\mathbb{R}^{n \times n}$ y $x\in\mathbb{R}^n$ entonces:
\begin{displaymath}
  O(Ax) = n(n \times mult + (n-1) \times sums) = (n^2)\times mult + n(n-1)\times sums = O(n^2)
\end{displaymath}
Si $y\in\mathbb{R}^n \Rightarrow \frac{y}{\parallel y \parallel} = O(n)$ por lo tanto reemplazando por lo nuevos valores: 
\begin{displaymath}
  O(metodoDeLaPotencia) = iter(O(n^2) + O(n)) = iterO(n^2) = O(n^2)
\end{displaymath}
porque $iter$ es un valor fijo que elegimos de antemano. La complejidad de aplicar deflacionar con
$v_i\in\mathbb{R}^n$ y $A\in\mathbb{R}^{n \times n}$ es:
\begin{displaymath}
  O(deflacionar) = O(A-\lambda{v_1}v_1^t) = n^2 \times restas + O(\lambda_i{v_i}v_i^t) 
\end{displaymath}
seguimos reemplazando:
\begin{displaymath}
n^2 \times restas + n^2 \times mults + O(v_i{v_i^t}) = n^2 \times restas + n^2 \times mults + n^2 \times mults
\end{displaymath}
entonces nos queda que:
\begin{displaymath}
 O(deflacionar) = n^2 \times restas + 2n^2 \times mults = O(n^2) 
\end{displaymath}
Por lo tanto la complejidad del método 1 queda de la siguiente forma:
\begin{displaymath}
O(metodo1) = k (O(metodoDeLaPotencia) + O(deflacionar)) = k(O(n^2) + O(n^2)) = O(n^2) 
\end{displaymath}
y la complejidad del método 2 es:
\begin{displaymath}
O(metodo2) = O(metodo1) + O(despejar)
\end{displaymath}
Y la complejidad de despejar la planteamos como:
\begin{displaymath}
O(despejar) = O(\frac{ \sqrt{\lambda_{i} } }{ \lambda_{i} }  A^{t}v)
\end{displaymath}
con $v_\in\mathbb{R}^n$ y $A\in\mathbb{R}^{n \times n}$ nos queda:
\begin{displaymath}
 n \times mults + O(A^{t}v) = n \times mults + (n(n \times mults + (n-1) \times sums))  
\end{displaymath}
lo que nos deja con:
\begin{displaymath}
 O(n) + nO(n) = O(n^2)
\end{displaymath}
reemplazando en la complejidad del método 2 obtenemos que:
\begin{displaymath}
 O(metodo2) = O(metodo1) + O(despejar) = O(n^2) + O(n^2) = O(n^2)
\end{displaymath}
Concluimos que el método 1 y el método 2 tienen la misma complejidad $O(n^2)$ tomando a $n$ como el tamaño de la matriz $A$ de entrada.
Como habíamos analizado en el comienzo el tamaño de la matriz $A$ que recibie cada uno es diferente. Para el método 1 es $n = pixels(img)$ y
para el método 2 es $n = nimgp \times \#personas$. Por lo tanto cuál de los dos cueste menos va a depender de la relación entre la resolución
de las imágenes de nuestra base de entrenamiento y la cantidad de imágenes que tenemos en total. En nuestro caso particular la resolución de las
imágenes excede ampliamente a la cantidad de imágenes. Nuestra base de entrenamiento tiene $410$ fotos y en general usamos una resolución
de $112 \times 92 = 10304$. Por lo tanto: 
\begin{displaymath}
 nimgp \times \#personas << pixels(img) \Rightarrow O(metodo2) << O(metodo1)
\end{displaymath}





