\subsection{Comparación de la complejidad de los métodos}
El método 1 trabaja con la matriz $AA^t$ y el método 2 trabaja con la matriz $A^tA$.
La matriz $A\in\mathbb{R}^{n \times m}$ con $m = pixels(img)$ y $n = nimgp * \#personas$. Por lo tanto $A^t\in\mathbb{R}^{m \times n}$.
Por lo tanto tenemos que $AA^t\in\mathbb{R}^{n \times n}$ y $A^tA\in\mathbb{R}^{m \times m}$.

En el \'unico punto que divergen los algoritmos es en el c\'alculo de los $k$ mayores autovalores, y sus
autovectores asociados, de $M_x = A^tA$, la matriz de covarianza. 
El m\'etodo 1 aplica el m\'etodo de la potencia con deflaci\'on directamente sobre la matriz $M_x$. 
El m\'etodo 2 primero aplica el m\'etodo de la potencia sobre la matriz $M^t_x = AA^t$ y después realiza un despeje para obtener los 
autovalores y autovectores de $M_x$. 
Por lo tanto tenemos que los dos métodos utilizan la misma función para calcular autovectores y autovalores:
\begin{algorithm}
\caption{calcularAutovaloresYAutovectores(Matriz $A$)}
\label{pseudo:calcularAutovaloresYAutovectores}
\begin{algorithmic}
\FOR {$i = 1$ hasta $k$}
  \STATE [$v_i$, $\lambda_i$] = metodoDeLaPotencia($A$)
  \STATE $A$ = deflacionar($A$, $v_i$, $\lambda_i$)
\ENDFOR
\end{algorithmic}
\end{algorithm}

y el método 2 además utiliza una función para despejar

\begin{algorithm}
\caption{despejar(Matriz $A$, Vector $v$, Double $\lambda$)}
\label{pseudo:despejar}
\begin{algorithmic}
  \STATE $u$ = $v\frac{ \sqrt{\lambda_{i} } }{ \lambda_{i} }  A^{t}$   
\end{algorithmic}
\end{algorithm}

por lo tanto la complejidad del método 1 es (abusando de la notación asintótica):

\begin{displaymath}
  O(m1) = k (O(metodoDeLaPotencia) + O(deflacionar))
\end{displaymath}

y del método 2:

\begin{displaymath}
  O(m1) = k (O(metodoDeLaPotencia) + O(deflacionar)) + k O(despejar)
\end{displaymath}

La complejidad del método de la potencia es:

\begin{displaymath}
  O(m1) = iter (O(Ax) + O(\frac{y}{{\parallel}y{\parallel}}))
\end{displaymath}

Si $A\in\mathbb{R}^{m \times n}$ y $x\in\mathbb{R}^n$ entonces:

\begin{displaymath}
  O(Ax) = m(n \times mult + (n-1) \times sums) = (mn)\times mult + m(n-1)\times sums = O(mn)
\end{displaymath}

Si $y\in
