\subsection{Comparación de la de la tasa de eficiencia en función de la cantidad de personas}
En este experimento se intenta analizar como influye la cantidad de personas en la tasa de eficiencia del reconocimiento de imágenes. Para esto vamos a concentrarnos solo en la variable de la cantidad de personas por lo que hay que fijar el resto de los parámetros en sus valores óptimos. La experimentación se va a hacer sobre la base de imágenes grandes, con cinco imágenes por persona (valor generoso pero dentro de los límites reales), y con quince componentes principales y con la técnica de reconocimiento más efectiva.

Se espera que cuantas más personas haya (y en consecuencia más imágenes), la probablidad de que haya imagénes similares de distintas personas o incluso personas similares es más alta, por lo que tasa de reconocimiento va a ser menor.
