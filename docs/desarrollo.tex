\section{Desarrollo}
%Deben explicarse los métodos numéricos que utilizaron y su aplicación al problema
%concreto involucrado en el trabajo práctico. Se deben mencionar los pasos que si-
%guieron para implementar los algoritmos, las dificultades que fueron encontrando y la
%descripción de cómo las fueron resolviendo. Explicar también cómo fueron planteadas
%y realizadas las mediciones experimentales. Los ensayos fallidos, hipótesis y conjeturas
%equivocadas, experimentos y métodos malogrados deben figurar en esta sección, con
%una breve explicación de los motivos de estas fallas (en caso de ser conocidas).
La implementación consta de 2 fases. Primero está la fase de entrenamiento:
\begin{enumerate}
 \item se obtiene la matriz $I\in\mathbb{R}^{n \times m}$ a partir de todas las imágenes de la base de entrenamiento. Para $i = 1,...,n$, llamamos $x_i$ a
 la i-ésima fila de $I$ que se corresponde con la i-ésima imagen de nuestra base vectorizada por filas. Cada posición de $x_i$ se 
 corresponde con un pixel de la imagen a la cual corresponde. Además, sabemos que $n = nimgp * \#sujetos$, siendo $nimgp$ la cantidad de imágenes
 que tengo por cada sujeto y $\#sujetos$ la cantidad de sujetos distintos a los que pertencen las imágenes de mi base de entrenamiento, y 
 $m = alto * ancho$, siendo $alto$ y $ancho$ los respectivos anchos y altos de las imágenes medidos en pixeles.
 
 \item se calcular la matriz $A$ a partir de la matriz $I$. Sus dimensiones son iguales. Obtenemos el vector $\mu = (x_1+...+x_n)/n$ que es 
 el promedio de las imágenes. Para calcular $A$, en primer lugar restamos a cada una de las filas de $I$ el vector $\mu$. Obtenemos $I'$ que contiene en la
 i-ésima fila al vector $(x_i - \mu)^t$. En segundo lugar obtenemos $A$ como: $A = I' * \sqrt{n-1}$.
 
 \item calculamos la matriz de covarianzas $M_x = A^t*A$.
 
 \item aplicamos el método de la potencia con deflación $k$ veces sobre la matriz $M_x$. $k$ es el número de componentes principales que nos 
 interesa obtener. Conseguimos una matriz $B\in\mathbb{R}^{m \times k}$ cuya columna $b_i$ se corresponde con el autovector $a_i$ de la matriz
 $M_x$ asociado al autovalor $\lambda_i$ tal que $\lambda_1 < lambda_2 < ... < lambda_k \leq lambda_{k+1} \leq ... \leq lambda_m$. Esto es así
 por que el método de la potencia, si encuentra un autovalor, es el de mayor módulo. \cite[1]{burden}.
 
 \item aplicamos la transformación característica a todas las imágenes de nuestra base de entrenamiento. La transformación característica
 de $x_i$ está dada por \begin{displaymath}
			    tc(x_i) = ({b_1^t}x_i, {b_2^t}x_i,..., {b_k^t}x_i)\in\mathbb{R}^k
                        \end{displaymath}

\end{enumerate}

y luego está la fase de identificación en la que recibimos una nueva imagen e identificamos a qué sujeto pertenece:
\begin{enumerate}
 \item recibimos la imagen y la vectorizamos por filas de la misma forma que lo hicimos con las imágenes en nuestra base de entrenamiento
 en la primera fase y obtenemos $x_{identificar}$.
 
 \item calculamos su transformación característica de la misma manera que lo hicimos en la fase de entrenamiento y con la misma matriz $B$
  \begin{displaymath}
			    tc(x_{identificar}) = ({b_1^t}x_{identificar}, {b_2^t}x_{identificar},..., {b_k^t}x_{identificar})\in\mathbb{R}^k
  \end{displaymath}
 
 \item decidimos a cuál sujeto pertenece por alguno de los dos métodos que describiremos más abajo.

\end{enumerate}
Dichos autovalores son necesarios ya que al reducir el espacio, naturalmente se pierde información, EXPLICAR MEJOR ESTO: los autovalores de menor 
módulo 
contienente menos información y por lo tanto es más confiable descartalos sin perder efectividad al identificar las caras.
\newline
El método de la potencia arrastra error al calcular un nuevo autovalor, por lo que no resulta práctico utilizarlo para calcular muchos. Sin embargo,
en este caso en particular solo tomamos los $k$ mayores.
\newline
Elegimos como vector inicial para el método de la potencia $b_0 = (1,...,1)$. 
La técnica de deflación arrastra el error generado por el método de la potencia, haciendo que los últimos autovalores tengan un error de cálculo 
mayor  que los primeros. De todas maneras esto tampoco es un problema grave, ya que la idea de las componentes principales es hallar un número reducido de las mismas.

El método alternativo consiste en hallar una relación entre los autovalores y autovectores de $B$ $=$ $A^{t} A$, y los autovectores y autovalores de $C$ $=$ $A A^{t}$.
Veamos primero como es la matriz $C$.

$C$ = $U \Sigma V^t (U \Sigma V^t)^{t}$ $=$ $U \Sigma V^t V \Sigma^{t} U^{t} $ $=$ $U \Sigma \Sigma^{t} U^{t}$. 

De esta manera encontramos la forma $SVD$ de la matriz $C$. Y en consecuencia podemos intuir que $U^{t}$ contiene los autovectores de $C$ en sus columnas.

Pasemos a confirmar la hipótesis anterior, sabemos que $u_{i}$ $=$ $\frac{1}{ \theta_{i} } A v_{i}$ con $\theta_{i}$ $=$ $\sqrt{ \lambda_{i} }$, y que $A^{t} A v_{i}$ $=$ $\lambda_{i} v_{i}$, por lo tanto:

$C u_{i}$ = $A A^{t} u_{i}$ $=$ $\frac{1}{ \theta_{i} } A A^{t}  A v_{i}$ $=$ $ \frac{ 1 }{ \sqrt{ \lambda_{i} } } A  \lambda_{i} v_{i}$ $=$ $ \lambda_{i} \frac{ 1 }{ \sqrt{ \lambda_{i} } } A  v_{i}$ $=$ $\lambda_{i} u_{i}$.

De lo anterior se deduce directamente que $\lambda_{i}$ es el autovalor asociado al autovector $u_{i}$ para la matriz $C$.

Veamos cómo se puede hallar $v_{i}$ conociendo $u_{i}$:

$u_{i}$ $=$ $\frac{1}{ \theta_{i} } A v_{i}$, 

multiplicando por $A^{t}$ en ambos lados se obtiene:

$A^{t} u_{i}$ $=$ $ \frac{1}{ \theta_{i} } A^{t} A v_{i}$, 

que es igual a: 

$A^{t} u_{i}$ $=$ $\frac{\lambda_{i}}{ \theta_{i} }  v_{i}$ 

despejando y reemplazando $\theta_{i}$:

$\frac{ \sqrt{\lambda_{i} } }{ \lambda_{i} }  A^{t} u_{i}$ $=$ $v_{i}$.

De lo anterior podemos describir el siguiente método para hallar los autovectores de $B$ $=$ $A^{t} A$:
\begin{enumerate}
\item Calcular $C$ $=$ $A A^{t}$.
\item Mediante el método de la potencia hallar los autovectores y autovalores de $C$, notar que los autovalores coinciden para ambas matrices, por lo que si  se podía utilizar el método para $B$, también se puede hacer para $C$.
\item Para cada autovector $u_{i}$ hallado, lo multiplico por $\frac{ \sqrt{\lambda_{i} } }{ \lambda_{i} }  A^{t}$ para obtener $v_{i}$.
\end{enumerate}


\subsection{Tests}

\subsection{Comparación de la complejidad de los métodos}
El método 1 trabaja con la matriz $AA^t$ y el método 2 trabaja con la matriz $A^tA$.
La matriz $A\in\mathbb{R}^{n \times m}$ con $m = pixels(img)$ y $n = nimgp * \#personas$. Por lo tanto $A^t\in\mathbb{R}^{m \times n}$.
Por lo tanto tenemos que $AA^t\in\mathbb{R}^{n \times n}$ y $A^tA\in\mathbb{R}^{m \times m}$.

En el \'unico punto que divergen los algoritmos es en el c\'alculo de los $k$ mayores autovalores, y sus
autovectores asociados, de $M_x = A^tA$, la matriz de covarianza. 
El m\'etodo 1 aplica el m\'etodo de la potencia con deflaci\'on directamente sobre la matriz $M_x$. 
El m\'etodo 2 primero aplica el m\'etodo de la potencia sobre la matriz $M^t_x = AA^t$. Luego

\subsubsection{Cantidad de componentes princiaples vs tasa de efectividad}
El objetivo de la siguiente experimentación es analizar cómo afecta el número de componentes principales utilizado a la tasa de efectividad
que obtenemos al identificar sujeto. Cuantas más componentes principales tomamos más información tenemos. Sin embargo, los autovalores que tomamos
son siempre los de mayor valor absoluto, es decir, los que más información reflejan. Por lo tanto es de esperar que cada componente principal que agregamos
nos aporta menos información que el anterior. Además, el método de la potencia con deflación arrastra error, esto implica que la exactitud de los autovalores 
obtenidos es cada vez menor $exactitud(\lambda_i) > exactitud(\lambda_{i+1}$.
Para la experimentación vamos a mantener la cantidad de sujetos, $40$, y la resolución de las imágenes, $112 \times 92$, constantes. El análisis lo vamos
a realizar considerando componentes hasta 45 partiendo de una sola y aumentando de a 3. Los tests se correrán tomando $nimgp = 3, 6, 9$ caras de entrenamiento. Para cada
combinación de $k$ y $nimgp$
sujetos a identificar = restantes
10 al azar por cada componente principal y después tomo el promedio
