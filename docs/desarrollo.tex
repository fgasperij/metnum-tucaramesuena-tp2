\section{Desarrollo}
%Deben explicarse los métodos numéricos que utilizaron y su aplicación al problema
%concreto involucrado en el trabajo práctico. Se deben mencionar los pasos que si-
%guieron para implementar los algoritmos, las dificultades que fueron encontrando y la
%descripción de cómo las fueron resolviendo. Explicar también cómo fueron planteadas
%y realizadas las mediciones experimentales. Los ensayos fallidos, hipótesis y conjeturas
%equivocadas, experimentos y métodos malogrados deben figurar en esta sección, con
%una breve explicación de los motivos de estas fallas (en caso de ser conocidas).
\subsection{La implementación y algunas decisiones de diseño}
La implementación consta de 2 fases. Primero está la fase de entrenamiento:
\begin{enumerate}
 \item se obtiene la matriz $I\in\mathbb{R}^{n \times m}$ a partir de todas las imágenes de la base de entrenamiento. Para $i = 1,...,n$, llamamos $x_i$ a
 la i-ésima fila de $I$ que se corresponde con la i-ésima imagen de nuestra base vectorizada por filas. Cada posición de $x_i$ se 
 corresponde con un pixel de la imagen a la cual corresponde. Además, sabemos que $n = nimgp * \#sujetos$, siendo $nimgp$ la cantidad de imágenes
 que tengo por cada sujeto y $\#sujetos$ la cantidad de sujetos distintos a los que pertencen las imágenes de mi base de entrenamiento, y 
 $m = alto * ancho$, siendo $alto$ y $ancho$ los respectivos anchos y altos de las imágenes medidos en pixeles.
 
 \item se calcular la matriz $A$ a partir de la matriz $I$. Sus dimensiones son iguales. Obtenemos el vector $\mu = (x_1+...+x_n)/n$ que es 
 el promedio de las imágenes. Para calcular $A$, en primer lugar restamos a cada una de las filas de $I$ el vector $\mu$. Obtenemos $I'$ que contiene en la
 i-ésima fila al vector $(x_i - \mu)^t$. En segundo lugar obtenemos $A$ como: $A = I' * \sqrt{n-1}$.
 
 \item calculamos la matriz de covarianzas $M_x = A^t*A$. (Para este punto y el siguiente efectuamos un análisis alternativo que se puede ver en \ref{metodoAlternativo})
 
 \item aplicamos el método de la potencia con deflación $k$ veces sobre la matriz $M_x$. $k$ es el número de componentes principales que nos 
 interesa obtener. Conseguimos una matriz $B\in\mathbb{R}^{m \times k}$ cuya columna $b_i$ se corresponde con el autovector $a_i$ de la matriz
 $M_x$ asociado al autovalor $\lambda_i$ tal que $\lambda_1 < lambda_2 < ... < lambda_k \leq lambda_{k+1} \leq ... \leq lambda_m$. Esto es así
 por que el método de la potencia, si encuentra un autovalor, es el de mayor módulo. \cite{burden}.
 
 \item aplicamos la transformación característica a todas las imágenes de nuestra base de entrenamiento. La transformación característica
 de $x_i$ está dada por \begin{displaymath}
			    tc(x_i) = ({b_1^t}x_i, {b_2^t}x_i,..., {b_k^t}x_i)\in\mathbb{R}^k
                        \end{displaymath}
\end{enumerate}
y luego está la fase de identificación en la que recibimos una nueva imagen e identificamos a qué sujeto pertenece:
\begin{enumerate}
 \item recibimos la imagen y la vectorizamos por filas de la misma forma que lo hicimos con las imágenes en nuestra base de entrenamiento
 en la primera fase y obtenemos $x_{identificar}$.
 
 \item calculamos su transformación característica de la misma manera que lo hicimos en la fase de entrenamiento y con la misma matriz $B$
  \begin{displaymath}
			    tc(x_{identificar}) = ({b_1^t}x_{identificar}, {b_2^t}x_{identificar},..., {b_k^t}x_{identificar})\in\mathbb{R}^k
  \end{displaymath}
 
 \item decidimos a cuál sujeto pertenece por alguno de los dos métodos que describimos más adelante.
\end{enumerate}
Los métodos que implementamos para decidir a qué sujeto pertenece una imagen luego de haber calculado su transformación
característica son los siguientes:
\begin{description}
  \item[Distancia mínima] calculamos la distancia entre la transformación característica de la imagen que queremos identificar
  y todas las imágenes de nuestra base. Decidimos que la imagen nueva corresponde al sujeto al cual pertenece la imagen cuya transformación
  característica es la más cercana a la transformación característica de la nueva imagen. Es decir, con el de distancia mínima.
  \item[Distancia mínima al promedio] calculamos el promedio de las distancias entre la transformación característica de la imagen que 
  queremos identificar y las imágenes de cada uno de nuestros sujetos. De esta forma obtenemos una distancia promedio a los diferentes sujetos.
  Finalmente, decidimos que la imagen nueva corresponde al sujeto al cual pertenece el promedio de distancias más cercano 
  a la transformación característica de la nueva imagen.
\end{description}
Aclaramos que el método de la potencia que implementamos recibe como vector inicial $b_0 = (1,...,1)$.
\par
La técnica de deflación arrastra el error generado por el método de la potencia, haciendo que los últimos autovalores tengan un error 
de cálculo mayor que los primeros. Este problema es mitigado por el hecho de que sólo tomamos los $k$ mayores pero de todas formas 
probablemente quede reflejado en la experimentación. Además, tomamos $iter = 300$ en la implementación del método de la potencia, el
máximo para que los tiempos de cómputo se mantuvieran en un rango aceptable, para obtener mayor precisión en el cálculo de los autovalores
y así compensar el error arrastrado por la deflación.

\subsection{Elección conveniente de la matriz de covarianzas para imágenes grandes}

El método alternativo consiste en hallar una relación entre los autovalores y autovectores de $B = A^tA$, y los autovectores y autovalores de $C = A A^t$.
Veamos primero como se escribe la matriz $C$. Si consideramos la descomposición en valores singulares de la matriz $A$:
\begin{displaymath}
  C = U \Sigma V^t (U \Sigma V^t)^{t} = U \Sigma V^t V \Sigma^{t} U^{t}  = U \Sigma \Sigma^{t} U^{t}
\end{displaymath}
De esta manera encontramos la forma $SVD$ de la matriz $C$. Y en consecuencia podemos intuir que $U^t$ contiene los autovectores de $C$ en sus columnas.
Pasemos a confirmar la hipótesis anterior, sabemos que $u_{i} = \frac{1}{\theta_{i}}Av_{i}$ con $\theta_{i} = \sqrt{ \lambda_{i} }$, y que $A^{t} A v_{i} = \lambda_{i} v_{i}$, por lo tanto:
\begin{displaymath}
C u_{i} = A A^{t} u_{i} = \frac{1}{ \theta_{i} } A A^{t}  A v_{i} =  \frac{ 1 }{  \theta_{i} } A  \lambda_{i} v_{i} =  \lambda_{i} \frac{ 1 }{ \theta_{i} } A  v_{i} = \lambda_{i} u_{i}.
\end{displaymath}
De lo anterior se deduce directamente que $\lambda_{i}$ es el autovalor asociado al autovector $u_{i}$ para la matriz $C$.
Veamos cómo se puede hallar $v_{i}$ conociendo $u_{i}$:
\begin{displaymath}
  u_{i} = \frac{1}{ \theta_{i} } A v_{i}
\end{displaymath}
multiplicando por $A^{t}$ en ambos lados se obtiene:
\begin{displaymath}
  A^{t} u_{i} =  \frac{1}{ \theta_{i} } A^{t} A v_{i},  
\end{displaymath}
que es igual a: 
\begin{displaymath}
  A^{t} u_{i} = \frac{\lambda_{i}}{ \theta_{i} }  v_{i} 
\end{displaymath}
despejando y reemplazando $\theta_{i}$:
\begin{displaymath}
  \frac{ \sqrt{\lambda_{i} } }{ \lambda_{i} }  A^{t} u_{i} = v_{i}
\end{displaymath}
De lo anterior podemos describir el siguiente método para hallar los autovectores de $B = A^{t}A$:
\begin{enumerate}
\setlength{\itemindent}{0.2in}
\item Calcular $C = AA^{t}$.
\item Mediante el método de la potencia hallar los autovectores y autovalores de $C$, notar que los autovalores coinciden para ambas matrices, por lo que si  se podía utilizar el método para $B$, también se puede hacer para $C$.
\item Para cada autovector $u_{i}$ hallado, lo multiplico por $\frac{ \sqrt{\lambda_{i} } }{ \lambda_{i} }  A^{t}$ para obtener $v_{i}$.
\end{enumerate}



%\subsection{Elección conveniente de la matriz de covarianzas para imágenes grandes}
\label{metodoalternativo}
El método alternativo consiste en utilizar la matriz $A.A^t$ en lugar de la matriz $A^tA$ a la hora de calcular los autovectores de la transformación característica. 
\par
Esto resulta posible, si advertimos que ambas matrices poseen los mismos autovalores. Veamos la demostración:
\par
Sea $u_{i}$ autovector de $A.A^{t}$ asociado a $\lambda_{i}$, entonces tenemos que:
\begin{displaymath}
  (A.A^{t}).u_{i} = \lambda_{i}.u_{i},  
\end{displaymath}
\par
multiplicando por $A^{t}$ a ambos lados de la igualdad, resulta:
\begin{displaymath}
  A^{t}.(A.A^{t}).u_{i} = A^{t}\lambda_{i}.u_{i},  
\end{displaymath}

que por propiedad de matrices y escalares, resulta:
\begin{displaymath}
 (A^{t}.A).(A^{t}.u_{i}) = \lambda_{i}.(A^{t}.u_{i}),  
\end{displaymath}
Y observando esta última igualdad, notamos que llamando $v_{i}$ a $A^{t}.u_{i}$ entonces $v_{i}$ resulta autovector de $A^{t}.A$ asociado a $\lambda_{i}$.
\par
De lo anterior, se desprende que todo autovalor de $A.A^{t}$ también lo es de $A^{t}.A$ (y tomando $A=A^{t}$ obtendríamos la vuelta). Este resultado nos permite elegir el producto de matrices que viva en un espacio de menor dimensión, y calcular sobre él, el método de la potencia, sabiendo que hallaremos los mismos autovalores.
\par
Además, la última línea de la demostración también nos indica cuál será el método para calcular el autovector asociado: $v_{i}=A^{t}.u_{i}$.



\subsection{Comparación de la complejidad de los métodos}
El método 1 trabaja con la matriz $AA^t$ y el método 2 trabaja con la matriz $A^tA$.
La matriz $A\in\mathbb{R}^{n \times m}$ con $m = pixels(img)$ y $n = nimgp * \#personas$. Por lo tanto $A^t\in\mathbb{R}^{m \times n}$.
Por lo tanto tenemos que $AA^t\in\mathbb{R}^{n \times n}$ y $A^tA\in\mathbb{R}^{m \times m}$.

En el \'unico punto que divergen los algoritmos es en el c\'alculo de los $k$ mayores autovalores, y sus
autovectores asociados, de $M_x = A^tA$, la matriz de covarianza. 
El m\'etodo 1 aplica el m\'etodo de la potencia con deflaci\'on directamente sobre la matriz $M_x$. 
El m\'etodo 2 primero aplica el m\'etodo de la potencia sobre la matriz $M^t_x = AA^t$. Luego

\subsection{Algoritmos}

% MÉTODO DE LA POTENCIA
\begin{algorithm}[!h]
\caption{metodoDeLaPotencia(Matriz $A$, Vector $v$, Int $iter$)}
\label{pseudo:metodoDeLaPotencia}
\begin{algorithmic}
  \FOR {$i=1$ hasta $iter$}
    \STATE $y = Ax$
    \STATE $x = \frac{y}{\parallel y \parallel}$
  \ENDFOR
\end{algorithmic}
\end{algorithm}
% DEFLACIÓN
\begin{algorithm}[!h]
\caption{deflacionar(Matriz $A$, Vector $autovector$, Double $\lambda$)}
\label{pseudo:deflacionar}
\begin{algorithmic}
  \STATE $V = vv^t$
  \STATE $B = A - \lambda V$
\end{algorithmic}
\end{algorithm}
% DESPEJE DE LOS AUTOS DE AtA TENIENDO LOS DE AAt
\begin{algorithm}[!h]
\caption{despejar(Matriz $A$, Vector $v$, Double $\lambda$)}
\label{pseudo:despejar}
\begin{algorithmic}
  \STATE $u$ = $v\frac{ \sqrt{\lambda_{i} } }{ \lambda_{i} }  A^{t}$   
\end{algorithmic}
\end{algorithm}
% IDENTIFICAR CARA - DISTANCIA MÍNIMA
\begin{algorithm}[!h]
\caption{identificarDistanciaMinima(Vector $tc_{nueva}$, Matriz $TC$)}
\label{pseudo:identificarDistanciMinima}
\begin{algorithmic}
  \FOR {$i=1$ hasta $\#filas(TC)$}
    \STATE $tc_{base}$ = $fila(i, TC)$
    \STATE $diff$ = $tc_{nueva}$ - $tc_{base}$
    \STATE $dist$ = $norm(diff)$
    \IF{ $i == 1$ }
      \STATE $minimo$ = $dist$
    \ENDIF
    \IF{ $dist < minimo$ }
      \STATE $minimo$ = $dist$
    \ENDIF    
  \ENDFOR
\end{algorithmic}
\end{algorithm}
% IDENTIFICAR CARA - DISTANCIA EN PROMEDIO MÍNIMA
\begin{algorithm}[!h]
\caption{identificarDistanciaPromedioMinima(Vector $tc_{nueva}$, Matriz $TC$, Int $iSujetos$, Int $nimgp$)}
\label{pseudo:identificarDistanciaPromedioMinima}
\begin{algorithmic}
  \FOR {$i=1$ hasta $iSujetos$}
    \STATE $acum$ = $0$
    \FOR {$j=1$ hasta $nimgp$}    
      \STATE $tc_{base}$ = $fila(nimgp*i+j, TC)$
      \STATE $diff$ = $tc_{nueva}$ - $tc_{base}$
      \STATE $acum$ += $norm(diff)$      
    \ENDFOR
    \STATE $iDistanciaPromedio$ = $\frac{acum}{nimgp}$
    \STATE agregar($iDistanciaPromedio$, $aDist$)
  \ENDFOR
  \STATE $sujeto$ = min($aDist$)
\end{algorithmic}
\end{algorithm}
\FloatBarrier
\subsection{Tests}
En esta sección haremos una introducción a las experimentaciones que decidimos realizar.
\subsection{Cantidad de componentes princiaples VS tasa de efectividad}
El objetivo de la siguiente experimentación es analizar como afecta el n\'umero de componentes principales utilizado a la tasa de efectividad
que obtenemos al identificar sujetos. Las variables a considerar son: 
personas = 40
sujetos a identificar = 1 imagen por persona
resolución = 112x92
componentes= 1-19 +3
5 al azar por cada componente principal y después tomo el promedio

\subsection{Comparación de la de la tasa de eficiencia en función de la cantidad de personas}
En este experimento se intenta analizar como influye la cantidad de personas en la tasa de eficiencia del reconocimiento de imágenes. Para esto vamos a concentrarnos solo en la variable de la cantidad de personas por lo que hay que fijar el resto de los parámetros en sus valores óptimos. La experimentación se va a hacer sobre la base de imágenes grandes, con cinco imágenes por persona (valor generoso pero dentro de los límites reales), y con quince componentes principales y con la técnica de reconocimiento más efectiva.

Se espera que cuantas más personas haya (y en consecuencia más imágenes), la probablidad de que haya imagénes similares de distintas personas o incluso personas similares es más alta, por lo que tasa de reconocimiento va a ser menor.

\subsubsection{nimgp vs Tasa de efectividad}
En la siguiente experimentación queremos analizar cómo afecta a la efectividad la cantidad de imágenes que tomamos por sujeto, $nimgp$, para la base
de entrenamiento a la tasa de efectividad. La base total con la que contamos contiene 41 sujetos y 10 fotos por cada uno de ellos. Es una base que
consideramos pequeña por lo cual vamos a intentar abarcar la mayor cantidad de combinaciones posibles de $nimgp$ y las fotos que tomamos de
entrenamiento con las de test.
¿Cómo vamos a variar las variables?
¿Qué esperamos obtener?
¿Por qué es interesante realizar este test?
¿Qué desventajas tenemos con la base dada?

personas = 40
sujetos a identificar = restantes
resolución = 112x92
componentes= 15, 30, 45
20 al azar por cada cantidad de caras y después tomo el promedio
caras que tomo por persona=1,9
\subsubsection{Resolución de las imágenes vs Tasa de efectividad}
En esta experimentación se analizará la tasa de efectividad en función de la la resolución de las imágenes utilizadas. Para esto se utilizaron 5 escenarios posibles que corresponden a diferentes configuraciones de los parámetros. De esta manera se definen las distintas condiciones del experimento en:
\begin{itemize}
\item Peśima: Se utilizan 10 componentes y 1 imagen por persona.
\item Mala: Se utilizan 10 componentes y 3 imágenes por persona.
\item Regular: Se utilizan 15 componentes y 5 imágenes por persona.
\item Buena: Se utilizan 30 componentes y 7 imágenes por persona.
\item Excelente: Se utilizan 30 componentes y 9 imágenes por persona.
\end{itemize}
La cantidad de personas se fijó en $41$ ya que este parámetro no modifica la condición de los escenarios tanto como los otros dos. Se realizaron varias experimentaciones para cada escencario y para cada resolución tomando imágenes al azar para cada sujeto. El método de identificación utilizado fue el de distancia mínima por que es el más efectivo. Este experimento intenta probar como afecta la resolución de las imágenes a la tasa de efectividad. Es natural pensar que cuanto mayor sea la resolución más información tendrán las imágenes por lo que resultados serán más exactos.

\subsubsection{Comparación de los métodos de identificación}
La siguiente experimentación tiene por objetivo comparar la tasa de efectividad entre el método de identificación por distancia mínima
y el método de identificación por distancia promedio mínima. Además, nos interesa determinar si la combinación óptima, en términos de 
la tasa de efectividad obtenida, es la misma para los dos métodos de identificación. Correremos los tests con las mismas variables que
en la primera experimentación pero aplicando el método de identificación por distancia promedio mínima para la fase de identificación.

