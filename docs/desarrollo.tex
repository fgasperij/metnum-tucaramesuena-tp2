\section{Desarrollo}
%Deben explicarse los métodos numéricos que utilizaron y su aplicación al problema
%concreto involucrado en el trabajo práctico. Se deben mencionar los pasos que si-
%guieron para implementar los algoritmos, las dificultades que fueron encontrando y la
%descripción de cómo las fueron resolviendo. Explicar también cómo fueron planteadas
%y realizadas las mediciones experimentales. Los ensayos fallidos, hipótesis y conjeturas
%equivocadas, experimentos y métodos malogrados deben figurar en esta sección, con
%una breve explicación de los motivos de estas fallas (en caso de ser conocidas).
El principal cálculo del análisis de las componentes principales es el correspondiente a hallar los autovalores de mayor módulo y sus autovectores asociados. El mismo se realiza con el método de la potencia que encuentra el autovalor de mayor módulo y luego se combina con la técnica de deflación [AGREGAR CITA - BURDEN] para ir encontrando el resto de los autovalores iterativamente. Este método tiene dos inconvenienttes, el primero es que requiere un vector inicial que debe cumplir ciertas condiciones que no se pueden conocer de antemano, por suerte, la probabilidad de elegir tal clase de vector es muy baja por lo que puede omitirse. En segundo lugar, hay que estudiar el error y la velocidad de convergencia para deducir el valor óptimo de la cantidad de iteraciones del algoritmo, es decir un valor númericamente aceptable que no induzca a errores. en un tiempo no prohibitivo. Por otro lado, la técnica de deflación arrastra el error generado por el método de la potencia, haciendo que los últimos autovalores tengan un error de cálculo mayor  que los primeros, limitando la cantidad de autovalores (y en consecuencia componentes principales) a calcular. De todas maneras esto tampoco es un problema grave, ya que la idea de las componentes principales es hallar un número reducido de las mismas (las mas importantes).

\subsection{Elección conveniente de la matriz de covarianzas para imágenes grandes}

El método alternativo consiste en hallar una relación entre los autovalores y autovectores de $B = A^tA$, y los autovectores y autovalores de $C = A A^t$.
Veamos primero como se escribe la matriz $C$. Si consideramos la descomposición en valores singulares de la matriz $A$:
\begin{displaymath}
  C = U \Sigma V^t (U \Sigma V^t)^{t} = U \Sigma V^t V \Sigma^{t} U^{t}  = U \Sigma \Sigma^{t} U^{t}
\end{displaymath}
De esta manera encontramos la forma $SVD$ de la matriz $C$. Y en consecuencia podemos intuir que $U^t$ contiene los autovectores de $C$ en sus columnas.
Pasemos a confirmar la hipótesis anterior, sabemos que $u_{i} = \frac{1}{\theta_{i}}Av_{i}$ con $\theta_{i} = \sqrt{ \lambda_{i} }$, y que $A^{t} A v_{i} = \lambda_{i} v_{i}$, por lo tanto:
\begin{displaymath}
C u_{i} = A A^{t} u_{i} = \frac{1}{ \theta_{i} } A A^{t}  A v_{i} =  \frac{ 1 }{  \theta_{i} } A  \lambda_{i} v_{i} =  \lambda_{i} \frac{ 1 }{ \theta_{i} } A  v_{i} = \lambda_{i} u_{i}.
\end{displaymath}
De lo anterior se deduce directamente que $\lambda_{i}$ es el autovalor asociado al autovector $u_{i}$ para la matriz $C$.
Veamos cómo se puede hallar $v_{i}$ conociendo $u_{i}$:
\begin{displaymath}
  u_{i} = \frac{1}{ \theta_{i} } A v_{i}
\end{displaymath}
multiplicando por $A^{t}$ en ambos lados se obtiene:
\begin{displaymath}
  A^{t} u_{i} =  \frac{1}{ \theta_{i} } A^{t} A v_{i},  
\end{displaymath}
que es igual a: 
\begin{displaymath}
  A^{t} u_{i} = \frac{\lambda_{i}}{ \theta_{i} }  v_{i} 
\end{displaymath}
despejando y reemplazando $\theta_{i}$:
\begin{displaymath}
  \frac{ \sqrt{\lambda_{i} } }{ \lambda_{i} }  A^{t} u_{i} = v_{i}
\end{displaymath}
De lo anterior podemos describir el siguiente método para hallar los autovectores de $B = A^{t}A$:
\begin{enumerate}
\setlength{\itemindent}{0.2in}
\item Calcular $C = AA^{t}$.
\item Mediante el método de la potencia hallar los autovectores y autovalores de $C$, notar que los autovalores coinciden para ambas matrices, por lo que si  se podía utilizar el método para $B$, también se puede hacer para $C$.
\item Para cada autovector $u_{i}$ hallado, lo multiplico por $\frac{ \sqrt{\lambda_{i} } }{ \lambda_{i} }  A^{t}$ para obtener $v_{i}$.
\end{enumerate}



\subsection{Tests}
En esta sección haremos una introducción a las experimentaciones que decidimos realizar.
\subsection{Cantidad de componentes princiaples VS tasa de efectividad}
El objetivo de la siguiente experimentación es analizar como afecta el n\'umero de componentes principales utilizado a la tasa de efectividad
que obtenemos al identificar sujetos. Las variables a considerar son: 
personas = 40
sujetos a identificar = 1 imagen por persona
resolución = 112x92
componentes= 1-19 +3
5 al azar por cada componente principal y después tomo el promedio

\subsection{Comparación de la de la tasa de eficiencia en función de la cantidad de personas}
En este experimento se intenta analizar como influye la cantidad de personas en la tasa de eficiencia del reconocimiento de imágenes. Para esto vamos a concentrarnos solo en la variable de la cantidad de personas por lo que hay que fijar el resto de los parámetros en sus valores óptimos. La experimentación se va a hacer sobre la base de imágenes grandes, con cinco imágenes por persona (valor generoso pero dentro de los límites reales), y con quince componentes principales y con la técnica de reconocimiento más efectiva.

Se espera que cuantas más personas haya (y en consecuencia más imágenes), la probablidad de que haya imagénes similares de distintas personas o incluso personas similares es más alta, por lo que tasa de reconocimiento va a ser menor.

\subsubsection{nimgp vs Tasa de efectividad}
En la siguiente experimentación queremos analizar cómo afecta a la efectividad la cantidad de imágenes que tomamos por sujeto, $nimgp$, para la base
de entrenamiento a la tasa de efectividad. La base total con la que contamos contiene 41 sujetos y 10 fotos por cada uno de ellos. Es una base que
consideramos pequeña por lo cual vamos a intentar abarcar la mayor cantidad de combinaciones posibles de $nimgp$ y las fotos que tomamos de
entrenamiento con las de test.
¿Cómo vamos a variar las variables?
¿Qué esperamos obtener?
¿Por qué es interesante realizar este test?
¿Qué desventajas tenemos con la base dada?

personas = 40
sujetos a identificar = restantes
resolución = 112x92
componentes= 15, 30, 45
20 al azar por cada cantidad de caras y después tomo el promedio
caras que tomo por persona=1,9
\subsubsection{Resolución de las imágenes vs Tasa de efectividad}
En esta experimentación se analizará la tasa de efectividad en función de la la resolución de las imágenes utilizadas. Para esto se utilizaron 5 escenarios posibles que corresponden a diferentes configuraciones de los parámetros. De esta manera se definen las distintas condiciones del experimento en:
\begin{itemize}
\item Peśima: Se utilizan 10 componentes y 1 imagen por persona.
\item Mala: Se utilizan 10 componentes y 3 imágenes por persona.
\item Regular: Se utilizan 15 componentes y 5 imágenes por persona.
\item Buena: Se utilizan 30 componentes y 7 imágenes por persona.
\item Excelente: Se utilizan 30 componentes y 9 imágenes por persona.
\end{itemize}
La cantidad de personas se fijó en $41$ ya que este parámetro no modifica la condición de los escenarios tanto como los otros dos. Se realizaron varias experimentaciones para cada escencario y para cada resolución tomando imágenes al azar para cada sujeto. El método de identificación utilizado fue el de distancia mínima por que es el más efectivo. Este experimento intenta probar como afecta la resolución de las imágenes a la tasa de efectividad. Es natural pensar que cuanto mayor sea la resolución más información tendrán las imágenes por lo que resultados serán más exactos.

\subsubsection{Comparación de los métodos de identificación}
La siguiente experimentación tiene por objetivo comparar la tasa de efectividad entre el método de identificación por distancia mínima
y el método de identificación por distancia promedio mínima. Además, nos interesa determinar si la combinación óptima, en términos de 
la tasa de efectividad obtenida, es la misma para los dos métodos de identificación. Correremos los tests con las mismas variables que
en la primera experimentación pero aplicando el método de identificación por distancia promedio mínima para la fase de identificación.

