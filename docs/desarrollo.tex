\section{Desarrollo}
%Deben explicarse los métodos numéricos que utilizaron y su aplicación al problema
%concreto involucrado en el trabajo práctico. Se deben mencionar los pasos que si-
%guieron para implementar los algoritmos, las dificultades que fueron encontrando y la
%descripción de cómo las fueron resolviendo. Explicar también cómo fueron planteadas
%y realizadas las mediciones experimentales. Los ensayos fallidos, hipótesis y conjeturas
%equivocadas, experimentos y métodos malogrados deben figurar en esta sección, con
%una breve explicación de los motivos de estas fallas (en caso de ser conocidas).
El principal cálculo del análisis de las componentes principales es el correspondiente a hallar los autovalores de mayor módulo y sus autovectores asociados. El mismo se realiza con el método de la potencia que encuentra el autovalor de mayor módulo y luego se combina con la técnica de deflación [AGREGAR CITA - BURDEN] para ir encontrando el resto de los autovalores iterativamente. Este método tiene dos inconvenienttes, el primero es que requiere un vector inicial que debe cumplir ciertas condiciones que no se pueden conocer de antemano, por suerte, la probabilidad de elegir tal clase de vector es muy baja por lo que puede omitirse. En segundo lugar, hay que estudiar el error y la velocidad de convergencia para deducir el valor óptimo de la cantidad de iteraciones del algoritmo, es decir un valor númericamente aceptable que no induzca a errores. en un tiempo no prohibitivo. Por otro lado, la técnica de deflación arrastra el error generado por el método de la potencia, haciendo que los últimos autovalores tengan un error de cálculo mayor  que los primeros, limitando la cantidad de autovalores (y en consecuencia componentes principales) a calcular. De todas maneras esto tampoco es un problema grave, ya que la idea de las componentes principales es hallar un número reducido de las mismas (las mas importantes).

El método alternativo consiste en hallar una relación entre los autovalores y autovectores de $B$ $=$ $A^{t} A$, y los autovectores y autovalores de $C$ $=$ $A A^{t}$.
Veamos primero como es la matriz $C$.

$C$ = $U \Sigma V^t (U \Sigma V^t)^{t}$ $=$ $U \Sigma V^t V \Sigma^{t} U^{t} $ $=$ $U \Sigma \Sigma^{t} U^{t}$. 

De esta manera encontramos la forma $SVD$ de la matriz $C$. Y en consecuencia podemos intuir que $U^{t}$ contiene los autovectores de $C$ en sus columnas.

Pasemos a confirmar la hipótesis anterior, sabemos que $u_{i}$ $=$ $\frac{1}{ \theta_{i} } A v_{i}$ con $\theta_{i}$ $=$ $\sqrt{ \lambda_{i} }$, y que $A^{t} A v_{i}$ $=$ $\lambda_{i} v_{i}$, por lo tanto:

$C u_{i}$ = $A A^{t} u_{i}$ $=$ $\frac{1}{ \theta_{i} } A A^{t}  A v_{i}$ $=$ $ \frac{ 1 }{ \sqrt{ \lambda_{i} } } A  \lambda_{i} v_{i}$ $=$ $ \lambda_{i} \frac{ 1 }{ \sqrt{ \lambda_{i} } } A  v_{i}$ $=$ $\lambda_{i} u_{i}$.

De lo anterior se deduce directamente que $\lambda_{i}$ es el autovalor asociado al autovector $u_{i}$ para la matriz $C$.

Veamos cómo se puede hallar $v_{i}$ conociendo $u_{i}$:

$u_{i}$ $=$ $\frac{1}{ \theta_{i} } A v_{i}$, 

multiplicando por $A^{t}$ en ambos lados se obtiene:

$A^{t} u_{i}$ $=$ $ \frac{1}{ \theta_{i} } A^{t} A v_{i}$, 

que es igual a: 

$A^{t} u_{i}$ $=$ $\frac{\lambda_{i}}{ \theta_{i} }  v_{i}$ 

despejando y reemplazando $\theta_{i}$:

$\frac{ \sqrt{\lambda_{i} } }{ \lambda_{i} }  A^{t} u_{i}$ $=$ $v_{i}$.

De lo anterior podemos describir el siguiente método para hallar los autovectores de $B$ $=$ $A^{t} A$:
\begin{enumerate}
\item Calcular $C$ $=$ $A A^{t}$.
\item Mediante el método de la potencia hallar los autovectores y autovalores de $C$, notar que los autovalores coinciden para ambas matrices, por lo que si  se podía utilizar el método para $B$, también se puede hacer para $C$.
\item Para cada autovector $u_{i}$ hallado, lo multiplico por $\frac{ \sqrt{\lambda_{i} } }{ \lambda_{i} }  A^{t}$ para obtener $v_{i}$.
\end{enumerate}


\subsection{Tests}

\subsection{Comparación de la complejidad de los métodos}
El método 1 trabaja con la matriz $AA^t$ y el método 2 trabaja con la matriz $A^tA$.
La matriz $A\in\mathbb{R}^{n \times m}$ con $m = pixels(img)$ y $n = nimgp * \#personas$. Por lo tanto $A^t\in\mathbb{R}^{m \times n}$.
Por lo tanto tenemos que $AA^t\in\mathbb{R}^{n \times n}$ y $A^tA\in\mathbb{R}^{m \times m}$.

En el \'unico punto que divergen los algoritmos es en el c\'alculo de los $k$ mayores autovalores, y sus
autovectores asociados, de $M_x = A^tA$, la matriz de covarianza. 
El m\'etodo 1 aplica el m\'etodo de la potencia con deflaci\'on directamente sobre la matriz $M_x$. 
El m\'etodo 2 primero aplica el m\'etodo de la potencia sobre la matriz $M^t_x = AA^t$. Luego

\subsubsection{Cantidad de componentes princiaples vs tasa de efectividad}
El objetivo de la siguiente experimentación es analizar cómo afecta el número de componentes principales utilizado a la tasa de efectividad
que obtenemos al identificar sujeto. Cuantas más componentes principales tomamos más información tenemos. Sin embargo, los autovalores que tomamos
son siempre los de mayor valor absoluto, es decir, los que más información reflejan. Por lo tanto es de esperar que cada componente principal que agregamos
nos aporta menos información que el anterior. Además, el método de la potencia con deflación arrastra error, esto implica que la exactitud de los autovalores 
obtenidos es cada vez menor $exactitud(\lambda_i) > exactitud(\lambda_{i+1}$.
Para la experimentación vamos a mantener la cantidad de sujetos, $40$, y la resolución de las imágenes, $112 \times 92$, constantes. El análisis lo vamos
a realizar considerando componentes hasta 45 partiendo de una sola y aumentando de a 3. Los tests se correrán tomando $nimgp = 3, 6, 9$ caras de entrenamiento. Para cada
combinación de $k$ y $nimgp$
sujetos a identificar = restantes
10 al azar por cada componente principal y después tomo el promedio
