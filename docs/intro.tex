\section{Introducción Teórica}
%Contendrá una breve explicación de la base teórica que fundamenta los métodos involu-
%crados en el trabajo, junto con los métodos mismos. No deben incluirse demostraciones
%de propiedades ni teoremas, ejemplos innecesarios, ni definiciones elementales (como
%por ejemplo la de matriz simétrica). En vez de definiciones básicas es conveniente citar
%ejemplos de bibliografía adecuada. Una cita vale más que mil palabras.
%
El objetivo principal de este informe es analizar el mecanismo de \emph{eigenfaces} para la identificación de personas 
usando solo sus caras en tiempo real con una tasa de efectividad aceptable. El mismo requiere de un mínimo de caras de 
entrenamiento iniciales (al menos una) por persona que se quiera identificar.
La idea central de la identificación de caras consiste en calcular la matriz de covarianza de las imagenes de entrenamiento. 
EXPLICAR PORQUE Y QUE ES LA MATRIZ DE COVARIANZA SIRVE, CERO DE PROBA. Una vez obtenida la matriz de covarianzas se hallan
las componentes principales de dicha matriz para reducir la dimensión del espacio en que trabajermos y reducir los tiempos 
de cómputo. Para esto es necesario hallar los autovalores de mayor módulo y sus autovectores asociados. Dichos autovalores 
son necesarios ya que al reducir el espacio, naturalmente se pierde información, EXPLICAR MEJOR ESTO: los autovalores de 
menor módulo contienen menos información y por lo tanto es más confiable descartalos sin perder efectividad al identificar 
las caras. Para obtener los autovalores aproximados (y sus respectivos autovectores) de la matriz de covarianzas se utiliza 
el método de la potencia con deflación. Este método arrastra error al calcular un nuevo autovalor, por lo que 
no resulta práctico utilizarlo para calcular todos los autovalores, afortunadamente, solamente estamos interesados en 
los primeros y más importantes.
Una vez obtenidos los autovectores y autovalores, podemos calcular las primeras componentes de imagen vectorizada para 
reducir la dimensión del espacio a trabajar.
Por último hay que identificar las caras que vayan apareciendo en el sistema, para esto se vectoriza y normaliza la 
imagen de la cara para posteriormente quedarnos con las primeras componentes de dicha imagen, y simplemente se utiliza 
algún criterio basado en  distancias entre la imagen a identificar y el resto de las imagenes.

