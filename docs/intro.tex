\section{Introducción Teórica}
%Contendrá una breve explicación de la base teórica que fundamenta los métodos involu-
%crados en el trabajo, junto con los métodos mismos. No deben incluirse demostraciones
%de propiedades ni teoremas, ejemplos innecesarios, ni definiciones elementales (como
%por ejemplo la de matriz simétrica). En vez de definiciones básicas es conveniente citar
%ejemplos de bibliografía adecuada. Una cita vale más que mil palabras.
%
El objetivo principal de este trabajo es implementar un método que nos permita identificar personas en tiempo real aplicando el 
análisis de componentes principales, conocido como PCA por sus siglas en inglés (Principal Component Analysis).\newline
Partimos de una base de datos de fotos, cada una asociada a un sujeto en particular. El objetivo es recibir una foto, que puede o no pertenecer a la base original, e identificar a qué sujeto pertenece.
\par
Se utilizó el enfoque propuesto en ``Eigenfaces for recognition'' \cite{eigenfaces}. El mismo propone trabajar
con las fotografías en 2 dimensiones sin reconstrucción geométrica de la tercera contando con un conjunto inicial -que 
llamaremos de entrenamiento- de imágenes pertenecientes al universo de rostros que se pretende reconocer. El enfoque propone trabajar 
con una transformación de ese espacio que permita reducir su dimensionalidad considerablemente sin perder la capacidad de distinguir 
a los elementos que viven en él.
