\documentclass[11pt, a4paper]{article}
\input{macrostp2.tex}
\usepackage{hyperref}
\usepackage{color}
\usepackage{paralist}
\usepackage{multirow}
\usepackage{footnote}
\makesavenoteenv{tabular}
\usepackage{pdflscape}
\usepackage{enumitem}
\usepackage{amsmath}
\usepackage{caption}
\usepackage{subcaption}
\usepackage{lscape} 

%El siguiente paquete permite escribir la caratula facilmente
\usepackage{caratula}
\hypersetup{
  pdftitle={ Métodos Numéricos - TP2 },
  colorlinks,
  citecolor=black,
  filecolor=black,
  linkcolor=black,
  urlcolor=black 
}

\newlength{\AnchoEncabezados}\setlength{\AnchoEncabezados}{6em}
\newcommand{\EncabezadoInline}[2]{
    
    \setlength{\hangindent}{\tadAnchoEncabezados + \parindent}%
    
        \parbox{\AnchoEncabezados}{\textbf{#1}}#2%
}

\newcommand{\Encabezado}[2]{%
    \par\EncabezadoInline{#1}{#2}\par%
}

\newcommand{\funcion}[2]{\textbf{#1}{#2}\BlankLine}

%\newcommand{\myFuncion}[3]{
%	\begin{algorithm}[H][H]
%	\dontprintsemicolon
%	{\bf {#1}} {#2}\;
%	\nocaptionofalgo
%	\Indp
%	{#3}
%	\Indm
%	\BlankLine
%	\end{algorithm}
%}
   
%Datos para la caratula
\materia{Métodos Numéricos}
\titulo{TP 2 - Tu cara me suena...}
\grupo{Grupo 2}
\integrante{Franco Bartalotta}{}{franco.bartalotta@hotmail.com}
\integrante{Fernando Gasperi Jabalera}{56/09}{fgasperijabalera@gmail.com}
\integrante{Ana Sarri\'es}{144/02}{abarloventos@gmail.com}

\resumen{En este trabajo práctico, desarrollamos un método para el reconocimiento de caras basado en el enfoque de \textit{eigenfaces}. Para ello, manipulamos la base de imágenes de entrenamiento matricialmente; obteniendo las componentes principales (autovectores) de su matriz de covarianzas a través del método de la potencia con deflación. Exploramos algunos métodos de clasificación, su tasa de efectividad y su relación con posibles variaciones en el conjunto inicial de entrenamiento.}
\claves{Face Recognition, Componentes Principales, SVD}

\begin{document}  
\maketitle
\tableofcontents

\newpage
\section{Introducción Teórica}
%Contendrá una breve explicación de la base teórica que fundamenta los métodos involu-
%crados en el trabajo, junto con los métodos mismos. No deben incluirse demostraciones
%de propiedades ni teoremas, ejemplos innecesarios, ni definiciones elementales (como
%por ejemplo la de matriz simétrica). En vez de definiciones básicas es conveniente citar
%ejemplos de bibliografía adecuada. Una cita vale más que mil palabras.
%
El objetivo principal de este trabajo es desarrollar un método que nos permita identificar personas en tiempo real a partir de sus rostros con el máximo grado de efectividad posible.
\newline
\par
\textbf{El enfoque utilizado es el propuesto en \cite{eigenfaces} cuya característica distintiva respecto de otros enfoques es que se basa en la teoría de la información.} Bajo el supuesto de que el reconocimiento facial puede hacerse a partir de fotografías 2-D, y contando con un conjunto inicial -que llamaremos de entrenamiento- de imágenes pertenecientes al universo de rostros que se pretende reconocer, el enfoque propone trabajar con una transformación de ese espacio que permita reducir su dimensionalidad considerablemente, conservando, a la vez, la capacidad de distinguir a los elementos que viven en él.
\newline
\par
Primero se calcula la matriz de covarianza de las imagenes de entrenamiento. EXPLICAR PORQUE Y QUE ES LA MATRIZ DE COVARIANZA SIRVE, CERO DE PROBA. Una vez obtenida la matriz de covarianzas se hallana las componentes principales de dicha matriz para reducir la dimensión del espacio en que trabajermos y reducir los tiempos de cómputo. Para esto es necesario hallar los autovalores de mayor módulo y sus autovectores asociados. Dichos autovalores son necesarios ya que al reducir el espacio, naturalmente se pierde información, EXPLICAR MEJOR ESTO: los autovalores de menor módulo contienente menos información y por lo tanto es más confiable descartalos sin perder efectividad al identificar las caras.
\newline
Para obtener los autovalores aproximados (y sus respectivos autovectores) de la matriz de covarianzas se utiliza el método de la potencia combinado con deflación. Este método arrastra error al calcular un nuevo autovalor, por lo que no resulta práctico utilizarlo para calcular todos los autovalores, afortunadamente, solamente estamos interesados en los primeros y más importantes.
Una vez obtenidos los autovectores y autovalores, podemos calcular las primeras componentes de imagen vectorizada para reducir la dimensión del espacio a trabajar.
\newline
Por último hay que identificar las caras que vayan apareciendo en el sistema, para esto se vectoriza y normaliza la imagen de la cara para posteriormente quedarnos con las primeras componentes de dicha imagen, y simplemente se utiliza algún criterio basado en  distancias entre la imagen a identificar y el resto de las imagenes.


\newpage
\section{Desarrollo}
%Deben explicarse los métodos numéricos que utilizaron y su aplicación al problema
%concreto involucrado en el trabajo práctico. Se deben mencionar los pasos que si-
%guieron para implementar los algoritmos, las dificultades que fueron encontrando y la
%descripción de cómo las fueron resolviendo. Explicar también cómo fueron planteadas
%y realizadas las mediciones experimentales. Los ensayos fallidos, hipótesis y conjeturas
%equivocadas, experimentos y métodos malogrados deben figurar en esta sección, con
%una breve explicación de los motivos de estas fallas (en caso de ser conocidas).
El principal cálculo del análisis de las componentes principales es el correspondiente a hallar los autovalores de mayor módulo y sus autovectores asociados. El mismo se realiza con el método de la potencia que encuentra el autovalor de mayor módulo y luego se combina con la técnica de deflación [AGREGAR CITA - BURDEN] para ir encontrando el resto de los autovalores iterativamente. Este método tiene dos inconvenienttes, el primero es que requiere un vector inicial que debe cumplir ciertas condiciones que no se pueden conocer de antemano, por suerte, la probabilidad de elegir tal clase de vector es muy baja por lo que puede omitirse. En segundo lugar, hay que estudiar el error y la velocidad de convergencia para deducir el valor óptimo de la cantidad de iteraciones del algoritmo, es decir un valor númericamente aceptable que no induzca a errores. en un tiempo no prohibitivo. Por otro lado, la técnica de deflación arrastra el error generado por el método de la potencia, haciendo que los últimos autovalores tengan un error de cálculo mayor  que los primeros, limitando la cantidad de autovalores (y en consecuencia componentes principales) a calcular. De todas maneras esto tampoco es un problema grave, ya que la idea de las componentes principales es hallar un número reducido de las mismas (las mas importantes).

El método alternativo consiste en hallar una relación entre los autovalores y autovectores de $B$ $=$ $A^{t} A$, y los autovectores y autovalores de $C$ $=$ $A A^{t}$.
Veamos primero como es la matriz $C$.

$C$ = $U \Sigma V^t (U \Sigma V^t)^{t}$ $=$ $U \Sigma V^t V \Sigma^{t} U^{t} $ $=$ $U \Sigma \Sigma^{t} U^{t}$. 

De esta manera encontramos la forma $SVD$ de la matriz $C$. Y en consecuencia podemos intuir que $U^{t}$ contiene los autovectores de $C$ en sus columnas.

Pasemos a confirmar la hipótesis anterior, sabemos que $u_{i}$ $=$ $\frac{1}{ \theta_{i} } A v_{i}$ con $\theta_{i}$ $=$ $\sqrt{ \lambda_{i} }$, y que $A^{t} A v_{i}$ $=$ $\lambda_{i} v_{i}$, por lo tanto:

$C u_{i}$ = $A A^{t} u_{i}$ $=$ $\frac{1}{ \theta_{i} } A A^{t}  A v_{i}$ $=$ $ \frac{ 1 }{ \sqrt{ \lambda_{i} } } A  \lambda_{i} v_{i}$ $=$ $ \lambda_{i} \frac{ 1 }{ \sqrt{ \lambda_{i} } } A  v_{i}$ $=$ $\lambda_{i} u_{i}$.

De lo anterior se deduce directamente que $\lambda_{i}$ es el autovalor asociado al autovector $u_{i}$ para la matriz $C$.

Veamos cómo se puede hallar $v_{i}$ conociendo $u_{i}$:

$u_{i}$ $=$ $\frac{1}{ \theta_{i} } A v_{i}$, 

multiplicando por $A^{t}$ en ambos lados se obtiene:

$A^{t} u_{i}$ $=$ $ \frac{1}{ \theta_{i} } A^{t} A v_{i}$, 

que es igual a: 

$A^{t} u_{i}$ $=$ $\frac{\lambda_{i}}{ \theta_{i} }  v_{i}$ 

despejando y reemplazando $\theta_{i}$:

$\frac{ \sqrt{\lambda_{i} } }{ \lambda_{i} }  A^{t} u_{i}$ $=$ $v_{i}$.

De lo anterior podemos describir el siguiente método para hallar los autovectores de $B$ $=$ $A^{t} A$:
\begin{enumerate}
\item Calcular $C$ $=$ $A A^{t}$.
\item Mediante el método de la potencia hallar los autovectores y autovalores de $C$, notar que los autovalores coinciden para ambas matrices, por lo que si  se podía utilizar el método para $B$, también se puede hacer para $C$.
\item Para cada autovector $u_{i}$ hallado, lo multiplico por $\frac{ \sqrt{\lambda_{i} } }{ \lambda_{i} }  A^{t}$ para obtener $v_{i}$.
\end{enumerate}


\subsection{Tests}

\subsection{Comparación de la complejidad de los métodos}
El método 1 trabaja con la matriz $AA^t$ y el método 2 trabaja con la matriz $A^tA$.
La matriz $A\in\mathbb{R}^{n \times m}$ con $m = pixels(img)$ y $n = nimgp * \#personas$. Por lo tanto $A^t\in\mathbb{R}^{m \times n}$.
Por lo tanto tenemos que $AA^t\in\mathbb{R}^{n \times n}$ y $A^tA\in\mathbb{R}^{m \times m}$.

En el \'unico punto que divergen los algoritmos es en el c\'alculo de los $k$ mayores autovalores, y sus
autovectores asociados, de $M_x = A^tA$, la matriz de covarianza. 
El m\'etodo 1 aplica el m\'etodo de la potencia con deflaci\'on directamente sobre la matriz $M_x$. 
El m\'etodo 2 primero aplica el m\'etodo de la potencia sobre la matriz $M^t_x = AA^t$. Luego

\subsubsection{Cantidad de componentes princiaples vs tasa de efectividad}
El objetivo de la siguiente experimentación es analizar cómo afecta el número de componentes principales utilizado a la tasa de efectividad
que obtenemos al identificar sujeto. Cuantas más componentes principales tomamos más información tenemos. Sin embargo, los autovalores que tomamos
son siempre los de mayor valor absoluto, es decir, los que más información reflejan. Por lo tanto es de esperar que cada componente principal que agregamos
nos aporta menos información que el anterior. Además, el método de la potencia con deflación arrastra error, esto implica que la exactitud de los autovalores 
obtenidos es cada vez menor $exactitud(\lambda_i) > exactitud(\lambda_{i+1}$.
Para la experimentación vamos a mantener la cantidad de sujetos, $40$, y la resolución de las imágenes, $112 \times 92$, constantes. El análisis lo vamos
a realizar considerando componentes hasta 45 partiendo de una sola y aumentando de a 3. Los tests se correrán tomando $nimgp = 3, 6, 9$ caras de entrenamiento. Para cada
combinación de $k$ y $nimgp$
sujetos a identificar = restantes
10 al azar por cada componente principal y después tomo el promedio

\newpage
\section{Resultados}
%Deben incluir los resultados de los experimentos, utilizando el formato más adecuado
%para su presentación. Deberón especificar claramente a qué experiencia corresponde
%cada resultado. No se incluirán aquí corridas de máquina. Algo fundamental en su
%aprendizaje en la materia es la presentación de resultados de forma clara y concisa para
%el lector.


\begin{figure}[H]{}
\centering
\includegraphics[scale=0.5]{graphs/CPvsTE.pdf}
\caption{Tasa de eficiencia en función de la cantidad de personas.}
\label{CPvsTE}
\end{figure}


\newpage
\section{Discusión}
%Se incluirá aquí un análisis de los resultados obtenidos en la sección anterior (se analizará
%su validez, coherencia, etc.). Deben analizarse como míınimo los ítems pedidos en el
%enunciado. No es aceptable decir que “los resultados fueron los esperados”, sin hacer
%clara referencia a la teoría la cual se ajustan. Además, se deben mencionar los resul-
%tados interesantes y los casos “patológicos” encontrados.

\subsection{Análisis de resultados de tasa de efectividad en functión de las componentes principales}
Se ve claramente como la tasa de efectividad aumenta al incrementarse el número de componentes principales utilizadas. Éste es un comportamiento
que esperábamos. Sin embargo, nos sorprende cuán rápido disminuye la velocidad a la que crece la tasa de efectividad. Para $k = 12$ podemos ver
que ya alcanzamos un $90\%$ de la tasa de efectividad total que conseguimos hasta donde alcanzaron nuestros tests, $k = 45$. Incrementar las 
componentes principales a partir de ese momento logra aumentar la tasa de efectividad sólo un $10\%$ más. Éste comportamiento interpretamos
que se debe a que las primeras 12 componentes principales sintetizan el $90\%$ de la información. Se desprende de dicha interpretación
el grado de importancia que tiene calcular los autovalores en orden descendente por magnitud de su módulo. Si tenemos en cuenta que la matriz
podría llegar a tener $10304$ autovalores diferentes, dada la resolución de las imágenes, las 12 componentes principales representan
menos de un $0,2\%$ de las mismas.
\par
A nivel práctico podemos aconsejar que a menos que los requerimientos sobre la tasa de efectividad sean muy demandantes basta con 
tomar unas pocas componentes principales para lograr una buena tasa de efectividad.
\par
Finalmente, podemos observar que la forma de las 3 curvas es prácticamente igual, simplemente sufren un desplazamiento sobre el eje
vertical. Esto nos lleva a intuir que tomar mayores $nimgp$ nos provee una mayor tasa de efectividad sin necesidad de aumentar el $k$ que
utilizamos.

\subsection{Análisis de la tasa de eficiencia en función de la cantidad de personas}
Como se puede observar en los resultados la tasa de efectividad disminuye a medida que aumenta la cantidad de personas, posiblemente debido a que, como se mencionó anteriormente, la probabilidad de que haya más sujetos parecidos entre sí o imágenes similares de distintos sujetos. Además la probabilidad de aciertos es menor al aumentar la cantidad de sujetos, es decir, si hay 2 sujetos la probabilidad de acierto sin ningún tipo de información de $\frac{1}{2}$, en cambio con 10 sujetos es $\frac{1}{10}$, en general la probablidad de acertar sin ningún tipo de información es $\frac{1}{n}$ donde $n$ es la cantidad de personas. Sería interesante analizar el comportamiento asintótico del gráfico para hallar una cota inferior para la tasa de eficiencia y deducir si el algoritmo puede ser arbitrariamente malo. Lamentablemente solo se dispone de $40$ imágenes, por lo que solo podemos dar una aproximación de la tasa de eficiencia para $40$ o menos imágenes.

\subsection{Análisis de la tasa de efectividad en función de $nimgp$}
Lo primero que nos llama la atención de estos resultados es la gran diferencia que devuelve en la tasa de efectividad que se obtuvo
tomando $nimgp = 1$ y variando $k = 15, 30, 45$. No estamos seguros a qué se debe este comportamiento. Intuimos que como sólo se toma
una foto por sujeto para la base de entrenamiento los resultados dependen mucho de la foto que se toma. Se intentó reducir este tipo
de ruido realizando varias veces el mismo test con seleccionando aleatoriamente las imágenes y aparentemente no tuvimos éxito para
el caso más sensible.
\par
Se puede apreciar claramente como la tasa de efectividad aumenta a medida que incrementamos $nimgp$, comportamiento que predijimos, sin embargo,
no nos esperábamos que luego de cierto umbral la tasa de efectividad comience a decrecer. Sin tener en cuenta el caso $nimgp = 1$, vemos 
claramente que la curva sigue una forma de parábola abriéndose hacia abajo, alcandando su máximo en $nimgp = 6$. No estamos seguros
a qué se debe el decrecimiento que sufre la tasa de efectividad para valores $nimgp > 6$.

\subsection{Análisis de la tasa de eficiencia en función de la resolución de las imágenes}
Los resultados obtenidos si bien no son los que se esperaban implican ciertas cuestiones interesantes. Primero, la resolución de imágenes son prácticamente idénticas para todas las condiciones por lo que se puede suponer que para ambas resoluciones la información que contienen las imágenes es suficiente y necesaria para que el análisis de componentes principales funcione de la manera esperada. Si bien nuestra suposición inicial no era correcta, es decir que a una resolución mayor se obtiene una tasa más alta, el error partió de suponer que la resolución más chica era lo suficiente mala como para arrojar peores resultados. Este percepción errónea provino de que las imágenes chicas son algo más difícil (pero no imposible con un poco de tiempo) de identificar a simple vista (con la cantidad apropiada a zoom).

\subsection{Análisis de comparación de los métodos de identificación}
Vemos que las curvas tienen la misma forma que en la primera experimentación, lo cual nos dice que los dos métodos de identificación
se comportan de la misma forma ante la variación de la cantidad de componentes principales utilizadas. Además, se comportan de la misma
forma ante los cambios en $nimgp$ ya que las 3 curvas aparecen una sobre la otra en el mismo orden en los dos gráficos. Sin embargo, 
podemos ver que para el método de identificación por distancia mínima las curvas roja y amarilla, correspondientes a $nimgp = 6$ y $nimgp = 9$
respectivamente, a partir de $k = 10$ obtienen tasas de efectividad superiores a $0.8$. En cambio, para el método de distancia promedio mínima
a penas superan la tasa de efectividad $0.8$. Concluimos que el método de identificación por distancia mínima alcanza una mayor tasa de 
efectividad utilizando la misma cantidad de componentes principales y valores de $nimgp$. Sugerimos utilizar éste método para cualquier
aplicación práctica por sobre el de distancia promedio mínima.


\newpage
\section{Conclusiones}
%Esta sección debe contener las conclusiones generales del trabajo. Se deben mencionar
%las relaciones de la discusión sobre las que se tiene certeza, junto con comentarios
%y observaciones generales aplicables a todo el proceso. Mencionar también posibles
%extensiones a los métodos, experimentos que hayan quedado pendientes, etc.
La aplicación de PCA ha demostrado ser una decisión acertada. Su poder quedó claro en la primera experimentación, tasa de efectividad en
función de las componentes principales, cuando tomando las componentes de mayor valor absoluto nos bastó con tomar menos de un $0,2\%$ de las
mismas para obtener hasta un $80\%$ de efectividad.
\par 
No pudimos apreciar cómo se comporta el arrastre de error de la deflación integrada al método de la potencia. Una experimentación posibles
para analizarlo sería ver qué pasa al aumentar el número de componenetes principales que se utiliza hasta valores cercanos al máximo posible.
La resolución de las imágenes debería ser reducida ya que de lo contrario el tiempo de ejecución puede llegar a valores poco prácticos.
\par
El método de la potencia con deflación se mostró práctico y útil, ya que con tan solo 300 iteraciones pudimos compensar en buena medida
el arrastre de error producido por la deflación. Esto lo intuimos a partir de que la tasa de efectividad al aumentar la cantidad de 
componentes principales en siempre continuó creciendo. Es decir, el error arrastrado fue compensado minimizado lo suficiente por 
la precisión obtenida con el método de la potencia como para no perturbar el crecimiento de la tasa de efectividad.
\par
Por último, nos gustaría destacar un detalle que nos resultó muy interesante, la elección de la matriz de covarianza. La diferencia 
abismal en complejidad que puede introducir dependiendo el contexto elegir una u otra forma de plantear los sistemas nos dejó una 
fuerte impresión. Nos agrada mucho ver el gran impacto que pueden tener en la práctica los contenidos teóricos desarrollados en la materia.

\newpage
\section{Apéndices}
\include{demo}
%En el apéndice A se incluirá el enunciado del TP. En el apéndice B se incluirán los
%códigos fuente de las funciones relevantes desde el punto de vista numérico. Resultados
%que valga la pena mencionar en el trabajo pero que sean demasiado específicos para
%aparecer en el cuerpo principal del trabajo podrán mencionarse en sucesivos apéndices
%rotulados con las letras mayusculas del alfabeto romano. Por ejemplo: la demostración
%de una propiedad que aplican para optimizar el algoritmo que programaron para resolver
%un problema.
\subsection{Descripción formal del enfoque de \textit{Eigenfaces for Recognition}}
\label{demo-formal-intro}
Formalmente, partimos de n imágenes (almacenada por filas en un vector) que notamos $x_i\in\mathbb{R}^m$ con $i = 1,...,n$. A cada uno de nuestros $x_i$ le restamos el vector $\mu = (x_1+...+x_n)/n$ para que tengan media igual cero. De esta forma construimos la matriz $X$ que
tiene en la i-ésima fila al vector $(x_i-\mu)^t$.  A partir de ésta, podemos construir la siguiente matriz $M_x$:
\begin{displaymath}
 M_x = \frac{1}{n-1}\hat{X^t}\hat{X}
\end{displaymath}
Esta matriz tiene la varianza de cada variable, que en nuestro caso son los pixeles de las imágenes, en la diagonal, y la covarianza entre ellos en las restantes posiciones, por lo que nos referimos a ella como la matriz de covarianzas. Esta matriz resulta ser simétrica y un teorema nos garantiza que toda matriz simétrica es diagonalizable en la forma $M_x = PDP^{-1}$ donde $D$ es una matriz diagonal.
Al diagonalizar, estamos buscamos variables que tengan covarianza cero entre sí y la 
mayor varianza posible. Si aplicamos como transformación un cambio de base apropiado:
\begin{displaymath}
 \hat{X^t} = PX^t
\end{displaymath}
$M_x$ nos queda:
\begin{displaymath}
 M_x = \frac{1}{n-1}\hat{X^t}\hat{X} = \frac{1}{n-1}(P\hat{X^t})(P\hat{X}) = PM_xP^t
\end{displaymath}
Como $M_x$ es simétrica existe $V$ ortogonal tal que $M_x = VD{V^t}$:
\begin{displaymath}
 M_x = PM_xP^t = P(VD{V^t})P^t = ({V^t}V)D(V{V^t}) = D
\end{displaymath}
habiendo tomado $P = V^t$. De esta forma vemos como las columnas $v_j$ de la matriz $V$ son los autovectores de $M_x$. Éstas van a ser
las componentes principales de nuestros datos, la base de imágenes de entrenamiento. En la práctica, en caso de que la dimensión 
de $M_x$ sea muy grande, es posible tomar sólo un subconjunto de las componentes principales, aquellas que capturen mayor proporción
de la varianza de los datos.
\par
De esta forma queda definida nuestra transformación característica por la aplicación del cambio de base a cada una de nuestras imágenes:
\begin{displaymath}
 tc(x_i) = \bar{V}^{t}x_i = (v_1^{t}x_i,...,v_k^{t}x_i)
\end{displaymath}
Ésta misma transformación característica se la aplicaremos a las imágenes que queramos identificar para evaluar a qué sujeto pertenecen.

\newpage
\subsection{Enunciado de la Cátedra}
\input{enunciado/tp2.tex}


\newpage
%\section{Referencias}
%Es importante incluir referencias a libros, artículos y páginas de Internet consultados
%durante el desarrollo del trabajo, haciendo referencia a estos materiales a lo largo del
%informe. Se deben citar también las comunicaciones personales con otros grupos.
\addcontentsline{toc}{section}{Referencias}
\begin{thebibliography}{1}
\bibitem{burden} Richard Burden. \textbf{Numerical Analysis.}  Brooks Cole, 2000.
\end{thebibliography}

\end{document}
