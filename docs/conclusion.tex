\section{Conclusiones}
%Esta sección debe contener las conclusiones generales del trabajo. Se deben mencionar
%las relaciones de la discusión sobre las que se tiene certeza, junto con comentarios
%y observaciones generales aplicables a todo el proceso. Mencionar también posibles
%extensiones a los métodos, experimentos que hayan quedado pendientes, etc.
La aplicación de PCA ha demostrado ser una decisión acertada. Su poder quedó claro en la primera experimentación, tasa de efectividad en
función de las componentes principales, cuando tomando las componentes de mayor valor absoluto nos bastó con tomar menos de un $0,2\%$ de las
mismas para obtener hasta un $80\%$ de efectividad.
\par 
No pudimos apreciar cómo se comporta el arrastre de error de la deflación integrada al método de la potencia. Una experimentación posibles
para analizarlo sería ver qué pasa al aumentar el número de componenetes principales que se utiliza hasta valores cercanos al máximo posible.
La resolución de las imágenes debería ser reducida ya que de lo contrario el tiempo de ejecución puede llegar a valores poco prácticos.
\par
El método de la potencia con deflación se mostró práctico y útil, ya que con tan solo 300 iteraciones pudimos compensar en buena medida
el arrastre de error producido por la deflación. Esto lo intuimos a partir de que la tasa de efectividad al aumentar la cantidad de 
componentes principales en siempre continuó creciendo. Es decir, el error arrastrado fue compensado minimizado lo suficiente por 
la precisión obtenida con el método de la potencia como para no perturbar el crecimiento de la tasa de efectividad.
\par
Por último, nos gustaría destacar un detalle que nos resultó muy interesante, la elección de la matriz de covarianza. La diferencia 
abismal en complejidad que puede introducir dependiendo el contexto elegir una u otra forma de plantear los sistemas nos dejó una 
fuerte impresión. Nos agrada mucho ver el gran impacto que pueden tener en la práctica los contenidos teóricos desarrollados en la materia.
