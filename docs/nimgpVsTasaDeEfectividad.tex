\subsubsection{nimgp vs Tasa de efectividad}
En la siguiente experimentación analizaremos cómo afecta a la tasa de efectividad la cantidad de imágenes que tomamos por sujeto, $nimgp$, 
para la base de entrenamiento a la tasa de efectividad. Nos interesa realizar este test porque tiene una relevancia grande en la práctica:
conocer qué valor $nimgp$ necesito como mínimo para poder alcanzar una tasa de efectividad aceptable para el contexto en el que lo aplique.
\par
La base total con la que contamos contiene 41 sujetos y 10 fotos por cada uno de ellos. 
La consideramos pequeña por lo cual vamos a intentar abarcar la mayor cantidad de combinaciones posibles entre $nimgp$, las fotos que tomamos 
de entrenamiento y las que tomamos para identificar test. Es por eso que vamos a tomar todos los $nimgp$ posibles, $nimgp = 1,..,9$, y para cada
uno de ellos vamos a correr 20 veces los tests eligiendo aleatoriamente las imágenes que pertenecen al conjunto de entrenamiento y las
que vamos a utilizar para la fase de identificación para luego quedarnos con el promedio de las tasas de efectividad.
El número de sujetos va a ser $40$, vamos a identificar todas las fotos que no pertenezcan a la base de entrenamiento, la resolución la
mantenemos en $112 \times 92$. El otro valor que variaremos para evaluar si afecta a la tasa obtenida para cada $nimgp$ es la cantidad de +
componentes principales, tomaremos $k = 15, 30, 45$.
\par
Esperamos que al tomar $nimgp$ mayores la tasa de efectividad aumente ya que estamos curbriendo más casos del rostro. Al cubrir un mayor
espectro de posibilidades del rostro estamos acotando la probabilidad de que la cara que nos presenten sea suficientemente distinta
como para que no podamos identificarla correctamente.