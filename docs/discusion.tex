\section{Discusión}
%Se incluirá aquí un análisis de los resultados obtenidos en la sección anterior (se analizará
%su validez, coherencia, etc.). Deben analizarse como míınimo los ítems pedidos en el
%enunciado. No es aceptable decir que “los resultados fueron los esperados”, sin hacer
%clara referencia a la teoría la cual se ajustan. Además, se deben mencionar los resul-
%tados interesantes y los casos “patológicos” encontrados.

\subsection{Análisis de resultados de tasa de efectividad en functión de las componentes principales}
Se ve claramente como la tasa de efectividad aumenta al incrementarse el número de componentes principales utilizadas. Éste es un comportamiento
que esperábamos. Sin embargo, nos sorprende cuán rápido disminuye la velocidad a la que crece la tasa de efectividad. Para $k = 12$ podemos ver
que ya alcanzamos un $90\%$ de la tasa de efectividad total que conseguimos hasta donde alcanzaron nuestros tests, $k = 45$. Incrementar las 
componentes principales a partir de ese momento logra aumentar la tasa de efectividad sólo un $10\%$ más. Éste comportamiento interpretamos
que se debe a que las primeras 12 componentes principales sintetizan el $90\%$ de la información. Se desprende de dicha interpretación
el grado de importancia que tiene calcular los autovalores en orden descendente por magnitud de su módulo. A nivel práctico podemos aconsejar
que a menos que los requerimientos sobre la tasa de efectividad sean muy demandantes basta con tomar unas pocas componentes principales
para lograr una buena tasa de efectividad.
\par
Finalmente, podemos observar que la forma de las 3 curvas es prácticamente igual, simplemente sufren un desplazamiento sobre el eje
vertical. Esto nos lleva a intuir que tomar mayores $nimgp$ nos provee una mayor tasa de efectividad sin necesidad de aumentar el $k$ que
utilizamos.

\subsection{Análisis de la tasa de eficiencia en función de la cantidad de personas}

\subsection{Análisis de la tasa de efectividad en función de $nimgp$}
Lo primero que nos llama la atención de estos resultados es la gran diferencia que devuelve en la tasa de efectividad que se obtuvo
tomando $nimgp = 1$ y variando $k = 15, 30, 45$. No estamos seguros a qué se debe este comportamiento. Intuimos que como sólo se toma
una foto por sujeto para la base de entrenamiento los resultados dependen mucho de la foto que se toma. Se intentó reducir este tipo
de ruido realizando varias veces el mismo test con seleccionando aleatoriamente las imágenes y aparentemente no tuvimos éxito para
el caso más sensible.
\par
Se puede apreciar claramente como la tasa de efectividad aumenta a medida que incrementamos $nimgp$, comportamiento que predijimos, sin embargo,
no nos esperábamos que luego de cierto umbral la tasa de efectividad comience a decrecer. Sin tener en cuenta el caso $nimgp = 1$, vemos 
claramente que la curva sigue una forma de parábola abriéndose hacia abajo, alcandando su máximo en $nimgp = 6$. No estamos seguros
a qué se debe el decrecimiento que sufre la tasa de efectividad para valores $nimgp > 6$.

\subsection{Análisis de comparación de los métodos de identificación}
Vemos que las curvas tienen la misma forma que en la primera experimentación, lo cual nos dice que los dos métodos de identificación
se comportan de la misma forma ante la variación de la cantidad de componentes principales utilizadas. Además, se comportan de la misma
forma ante los cambios en $nimgp$ ya que las 3 curvas aparecen una sobre la otra en el mismo orden en los dos gráficos. Sin embargo, 
podemos ver que para el método de identificación por distancia mínima las curvas roja y amarilla, correspondientes a $nimgp = 6$ y $nimgp = 9$
respectivamente, a partir de $k = 10$ obtienen tasas de efectividad superiores a $0.8$. En cambio, para el método de distancia promedio mínima
a penas superan la tasa de efectividad $0.8$. Concluimos que el método de identificación por distancia mínima alcanza una mayor tasa de 
efectividad utilizando la misma cantidad de componentes principales y valores de $nimgp$. Sugerimos utilizar éste método para cualquier
aplicación práctica por sobre el de distancia promedio mínima.

