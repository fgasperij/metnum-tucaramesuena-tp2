\subsubsection{Resolución de las imágenes vs Tasa de efectividad}
En esta experimentación se analizará la tasa de efectividad en función de la la resolución de las imágenes utilizadas. Para esto se utilizaron 5 escenarios posibles que corresponden a diferentes configuraciones de los parámetros. De esta manera se definen las distintas condiciones del experimento en:
\begin{itemize}
\item Peśima: Se utilizan 10 componentes y 1 imagen por persona.
\item Mala: Se utilizan 10 componentes y 3 imágenes por persona.
\item Regular: Se utilizan 15 componentes y 5 imágenes por persona.
\item Buena: Se utilizan 30 componentes y 7 imágenes por persona.
\item Excelente: Se utilizan 30 componentes y 9 imágenes por persona.
\end{itemize}
La cantidad de personas se fijó en $41$ ya que este parámetro no modifica la condición de los escenarios tanto como los otros dos. Se realizaron varias experimentaciones para cada escencario y para cada resolución tomando imágenes al azar para cada sujeto. El método de identificación utilizado fue el de distancia mínima.
