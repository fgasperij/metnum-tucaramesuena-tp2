\subsection{Elección conveniente de la matriz de covarianzas para imágenes grandes}

El método alternativo consiste en hallar una relación entre los autovalores y autovectores de $B = A^tA$, y los autovectores y autovalores de $C = A A^t$.
Veamos primero como se escribe la matriz $C$. Si consideramos la descomposición en valores singulares de la matriz $A$:
\begin{displaymath}
  C = U \Sigma V^t (U \Sigma V^t)^{t} = U \Sigma V^t V \Sigma^{t} U^{t}  = U \Sigma \Sigma^{t} U^{t}
\end{displaymath}
De esta manera encontramos la forma $SVD$ de la matriz $C$. Y en consecuencia podemos intuir que $U^t$ contiene los autovectores de $C$ en sus columnas.
Pasemos a confirmar la hipótesis anterior, sabemos que $u_{i} = \frac{1}{\theta_{i}}Av_{i}$ con $\theta_{i} = \sqrt{ \lambda_{i} }$, y que $A^{t} A v_{i} = \lambda_{i} v_{i}$, por lo tanto:
\begin{displaymath}
C u_{i} = A A^{t} u_{i} = \frac{1}{ \theta_{i} } A A^{t}  A v_{i} =  \frac{ 1 }{  \theta_{i} } A  \lambda_{i} v_{i} =  \lambda_{i} \frac{ 1 }{ \theta_{i} } A  v_{i} = \lambda_{i} u_{i}.
\end{displaymath}
De lo anterior se deduce directamente que $\lambda_{i}$ es el autovalor asociado al autovector $u_{i}$ para la matriz $C$.
Veamos cómo se puede hallar $v_{i}$ conociendo $u_{i}$:
\begin{displaymath}
  u_{i} = \frac{1}{ \theta_{i} } A v_{i}
\end{displaymath}
multiplicando por $A^{t}$ en ambos lados se obtiene:
\begin{displaymath}
  A^{t} u_{i} =  \frac{1}{ \theta_{i} } A^{t} A v_{i},  
\end{displaymath}
que es igual a: 
\begin{displaymath}
  A^{t} u_{i} = \frac{\lambda_{i}}{ \theta_{i} }  v_{i} 
\end{displaymath}
despejando y reemplazando $\theta_{i}$:
\begin{displaymath}
  \frac{ \sqrt{\lambda_{i} } }{ \lambda_{i} }  A^{t} u_{i} = v_{i}
\end{displaymath}
De lo anterior podemos describir el siguiente método para hallar los autovectores de $B = A^{t}A$:
\begin{enumerate}
\setlength{\itemindent}{0.2in}
\item Calcular $C = AA^{t}$.
\item Mediante el método de la potencia hallar los autovectores y autovalores de $C$, notar que los autovalores coinciden para ambas matrices, por lo que si  se podía utilizar el método para $B$, también se puede hacer para $C$.
\item Para cada autovector $u_{i}$ hallado, lo multiplico por $\frac{ \sqrt{\lambda_{i} } }{ \lambda_{i} }  A^{t}$ para obtener $v_{i}$.
\end{enumerate}


