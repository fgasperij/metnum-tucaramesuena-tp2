\subsection{Cantidad de componentes princiaples VS tasa de efectividad}
El objetivo de la siguiente experimentación es analizar como afecta el n\'umero de componentes principales utilizado a la tasa de efectividad
que obtenemos al identificar sujetos. Las variables a considerar son: 
personas = 40
sujetos a identificar = 1 imagen por persona
resolución = 112x92
componentes= 1-70 +3
5 al azar por cada componente principal y después tomo el promedio
3 caras, 6 caras, 9 caras

Con un método de identificación y con el otro.