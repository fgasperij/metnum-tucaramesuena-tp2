\subsubsection{Cantidad de componentes princiaples vs tasa de efectividad}
El objetivo de la siguiente experimentación es analizar cómo afecta el número de componentes principales utilizado a la tasa de efectividad
que obtenemos al identificar fotos. Cuantas más componentes principales tomamos más información tenemos. Sin embargo, los $k$ calculamos
autovalores que tomamos son los $k$ de mayor valor absoluto, es decir, los que más información reflejan. Por lo tanto es de esperar que 
cada componente principal que agregamos nos aporte menos información que la anterior. Es decir, que la tasa de efectividad aumente junto con la 
cantidad de componentes principales utilizadas pero a una velocidad sublineal. Además, sabemos que el método de la potencia con deflación arrastra 
error, esto implica que la exactitud de los autovalores obtenidos es cada vez menor $exactitud(\lambda_i) > exactitud(\lambda_{i+1}$.
Para la experimentación vamos a mantener la cantidad de sujetos, $40$, y la resolución de las imágenes, $112 \times 92$, constantes. 
El análisis lo vamos a realizar considerando hasta 45 componentes principales. Los tests se correrán tomando $nimgp = 3, 6, 9$ caras de 
entrenamiento. Para cada combinación de $k$ y $nimgp$ corremos el test 10 veces, tomando al azar las fotos de la base de entrenamiento y las 
utilizadas para identificación. Calculamos la tasa de efectividad para los 10 y nos quedamos con el promedio. Vamos a intentar identificar
a todas las fotos restantes de la base para poder tener más casos.