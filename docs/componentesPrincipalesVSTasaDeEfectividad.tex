\subsubsection{Cantidad de componentes princiaples vs tasa de efectividad}
El objetivo de la siguiente experimentación es analizar cómo afecta el número de componentes principales utilizado a la tasa de efectividad
que obtenemos al identificar sujeto. Cuantas más componentes principales tomamos más información tenemos. Sin embargo, los autovalores que tomamos
son siempre los de mayor valor absoluto, es decir, los que más información reflejan. Por lo tanto es de esperar que cada componente principal que agregamos
nos aporta menos información que el anterior. Además, el método de la potencia con deflación arrastra error, esto implica que la exactitud de los autovalores 
obtenidos es cada vez menor $exactitud(\lambda_i) > exactitud(\lambda_{i+1}$.
Para la experimentación vamos a mantener la cantidad de sujetos, $40$, y la resolución de las imágenes, $112 \times 92$, constantes. El análisis lo vamos
a realizar considerando componentes hasta 45 partiendo de una sola y aumentando de a 3. Los tests se correrán tomando $nimgp = 3, 6, 9$ caras de entrenamiento. Para cada
combinación de $k$ y $nimgp$
sujetos a identificar = restantes
10 al azar por cada componente principal y después tomo el promedio