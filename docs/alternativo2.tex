\subsection{Elección conveniente de la matriz de covarianzas para imágenes grandes}
\label{metodoalternativo}
El método alternativo consiste en utilizar la matriz $A.A^t$ en lugar de la matriz $A^tA$ a la hora de calcular los autovectores de la transformación característica. 
\par
Esto resulta posible, si advertimos que ambas matrices poseen los mismos autovalores. Veamos la demostración:
\par
Sea $u_{i}$ autovector de $A.A^{t}$ asociado a $\lambda_{i}$, entonces tenemos que:
\begin{displaymath}
  (A.A^{t}).u_{i} = \lambda_{i}.u_{i},  
\end{displaymath}
\par
multiplicando por $A^{t}$ a ambos lados de la igualdad, resulta:
\begin{displaymath}
  A^{t}.(A.A^{t}).u_{i} = A^{t}\lambda_{i}.u_{i},  
\end{displaymath}

que por propiedad de matrices y escalares, resulta:
\begin{displaymath}
 (A^{t}.A).(A^{t}.u_{i}) = \lambda_{i}.(A^{t}.u_{i}),  
\end{displaymath}
Y observando esta última igualdad, notamos que llamando $v_{i}$ a $A^{t}.u_{i}$ entonces $v_{i}$ resulta autovector de $A^{t}.A$ asociado a $\lambda_{i}$.
\par
De lo anterior, se desprende que todo autovalor de $A.A^{t}$ también lo es de $A^{t}.A$ (y tomando $A=A^{t}$ obtendríamos la vuelta). Este resultado nos permite elegir el producto de matrices que viva en un espacio de menor dimensión, y calcular sobre él, el método de la potencia, sabiendo que hallaremos los mismos autovalores.
\par
Además, la última línea de la demostración también nos indica cuál será el método para calcular el autovector asociado: $v_{i}=A^{t}.u_{i}$.


